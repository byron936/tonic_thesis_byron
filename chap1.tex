\chapter{Introduction}
\label{chap:introduction}

% Introduction should provide appropriate context and background for your research, such as the recent trend and importance of the technology development related to your thesis.

% NTN introduction
Non-terrestrial networks (NTNs) have emerged as a promising solution for next-generation communication systems. They enable network connectivity in areas that are beyond the reach of conventional terrestrial infrastructure. In particular, in natural environments such as forests, oceans, and deserts, terrestrial networks are not only impractical to deploy due to the sparse distribution of user equipment (UE) but also environmentally unsustainable, as their construction may cause significant ecological disturbance. A variety of platforms can be employed to provide network services in NTNs, including geostationary Earth orbit (GEO), medium Earth orbit (MEO), and low Earth orbit (LEO) satellites, as well as unmanned aerial vehicles (UAVs) and high-altitude drones. Among these, LEO satellite systems have attracted the most attention in recent research. Operating at altitudes ranging approximately from 500 km to 2000 km, LEO satellites offer global coverage comparable to GEO and MEO systems but with significantly shorter propagation distances. This feature leads to reduced latency and lower propagation loss, thereby improving the received signal-to-noise ratio (SNR) and enhancing overall data throughput performance.

% beamforming and multiple cells
Practically, each LEO satellite covers a vast service area to achieve global coverage. However, transmitting signals using a single beam encompassing the entire coverage area is impractical, as the resultant SNR at the ground UE would be insufficient. To ensure the required received SNR at the UE side, beamforming techniques are employed to concentrate the beam energy and illuminate specific spots, which correspond to cells within the overall coverage area. By employing this approach, the entire coverage area is effectively partitioned into tens or even hundreds of smaller cells. Consequently, a critical challenge arises regarding the optimal allocation of resources among these cells in order to maximize overall network performance and efficiency.

% existed NTN beamhopping work
Resource allocation in beam hopping has been widely studied as a key issue in LEO satellite communication networks. Beam hopping refers to the technique where active beams on a LEO satellite sequentially serve different cells during different time slots, thereby providing coverage to the entire service area. In this context, optimizing power allocation and beam pattern scheduling to maximize overall system throughput has been extensively explored~\cite{Joint-Power-Allocation}. Additionally, latency optimization through power allocation and beam scheduling has also garnered significant attention~\cite{Latency-Optimization}. However, to the best of our knowledge, existing research has not addressed the problem of initial access delay within the beam hopping framework.

% initial access delay
In the 3rd Generation Partnership Project (3GPP) 5G network architecture, Ultra-Reliable and Low-Latency Communications (URLLC) has been identified as a key service category, enabling latency-critical applications such as smart manufacturing, emergency response, and real-time media streaming. Although LEO satellite communication provides the shortest propagation delay among satellite orbit types compared to MEO and GEO systems, the overall end-to-end connection time between a satellite and a ground UE remains relatively large due to factors such as long transmission distances and complex link establishment procedures. Particularly in the initial cell search phase, the process is further complicated in LEO satellite communications due to the aforementioned cell structure, which impacts connection setup latency and resource management.

% TN beam selecting work
Several studies have investigated initial access delay in terrestrial networks, particularly in millimeter-wave scenarios, which are key technologies in New Radio (NR) and beyond networks. For example,~\cite{Directional-Reference-Signal} focuses on maximizing the number of reference signals received per time slot while ensuring fairness and limiting the number of time slots per access round. Another work~\cite{AI-Aided-Beam-Tracking} examines initial access delay in autonomous vehicle networks, identifying the main challenge as the selection of the transmit/receive (Tx/Rx) beam pair in terrestrial systems. However, the initial access search process in LEO satellite networks presents greater complexity due to the unique characteristics of satellite communication and the multi-cell beam structure, making the challenge more intricate than in terrestrial networks.

% Beam number and SSB power

There are two primary factors influencing the delay in the initial cell search phase in LEO satellite communications: the number of simultaneously active beams and the transmitted power of the synchronization signal block (SSB). Firstly, supporting multiple active beams enables a satellite to serve multiple cells concurrently. Combined with the beam hopping technique, these active beams sequentially illuminate different cells in distinct time slots to cover the entire service area. Implementing such multi-beam service requires deploying a large-scale phased antenna array onboard the satellite, whose physical size directly impacts both the cost and engineering complexity of the satellite design. Moreover, an increased number of simultaneously active beams means that the available power must be divided among more beams, linking closely to the second factor. The transmitted power of the SSB significantly affects the success rate of connection establishment between the satellite and user equipment (UE). Insufficient transmitted power may cause the UE to incorrectly decode the signal, leading to increased connection latency. Conversely, excessive transmitted power is inefficient since the SSB is a control signal; boosting its power does not yield proportional gains as it would for data transmission. Furthermore, because the satellite's total transmitted power mainly relies on onboard solar panels with limited energy supply, power must be managed efficiently to ensure reliable and sustainable operation.

% our problem

% However, to acheive LEO satellite network service, there are some key challanges that need to be resolved. One of the challanges is the unavoidable frequent handover~\cite{38821}. The high speed of LEO satellite forces user equipments (UEs) on the ground to switch the serving satellites frequently, as shown in Figure~\ref{time to ho}. In such scenario, the signalling overhead led by handover signals and the handover interruption time are big issues. 3GPP has discussed some solutions to deal with these issues. By using quasi-earth-fixed cell and satellite switch with re-synchronization~\cite{38300}, the frequent handovers are avoided and the signalling overhead is reduced. For UEs that have already accessed to the network, they can receive the upcoming serving satellite information from the previous one. Also, with the help of the ephemeris data of satellies and the position information of UEs, the matching between satellites and UEs has been largely improved~\cite{38331}. Nontheless, for those UEs who have not access to the network, the random access procedure is the only way for them to get into the network. To establish the connection between satellites and UEs, satellites need to transmit synchronization signal block (SSB) to ground for UEs to capture. Once UEs have successfully receive SSBs, the random access procedure starts and the UEs are able to access to the satellite network. Typically in terrestrial network, ground stations send SSB to the serving area every 20 milliseconds. However, the traditional SSB specifications in LEO satellite communication system do not work because the power budget in LEO satellite is so tight that the power is not enough to send all the serving cells in such a short periodicity. Thus, in this thesis we will find out how to deal with this issue by adjust the SSB periodicity and transmitted power. 

\begin{figure}[h!]
    \centering
    \includegraphics[width=1\textwidth]{figure/time to ho.pdf}
    \caption{Time to handover for min/max cell diameter and varying UE speed}
    \label{time to ho}
\end{figure}