\chapter{Introduction}
\label{chap:introduction}

% Introduction should provide appropriate context and background for your research, such as the recent trend and importance of the technology development related to your thesis.

Non-terrestrial networks (NTNs) have emerged as a promising solution for next-generation communication systems. They enable network connectivity in areas that are beyond the reach of conventional terrestrial infrastructure. In particular, in natural environments such as forests, oceans, and deserts, terrestrial networks are not only impractical to deploy due to the sparse distribution of user equipment (UE) but also environmentally unsustainable, as their construction may cause significant ecological disturbance. A variety of platforms can be employed to provide network services in NTNs, including geostationary Earth orbit (GEO), medium Earth orbit (MEO), and low Earth orbit (LEO) satellites, as well as unmanned aerial vehicles (UAVs) and high-altitude drones. Among these, LEO satellite systems have attracted the most attention in recent research. Operating at altitudes ranging approximately from 500 km to 2000 km, LEO satellites offer global coverage comparable to GEO and MEO systems but with significantly shorter propagation distances. This feature leads to reduced latency and lower propagation loss, thereby improving the received signal-to-noise ratio (SNR) and enhancing overall data throughput performance.

In the 3rd Generation Partnership Project (3GPP) 5G network architecture, Ultra-Reliable and Low-Latency Communications (URLLC) has been identified as a key service category, enabling latency-critical applications such as smart manufacturing, emergency response, and real-time media streaming. Although LEO satellite communication provides the shortest propagation delay among satellite orbit types compared to MEO and GEO systems, the overall end-to-end connection time between a satellite and a ground UE remains relatively large due to factors such as long transmission distances and complex link establishment procedures. 

Particularly in initial cell searching, it is further complicated in LEO satellite communication. In practical, each LEO satellite covers a huge service area so as to acheive global coverage. However, it is impractical to transmit message with one beam that covers the whole area since the SNR will be too low. To achieve the required received SNR at the ground UE side, beamforming technique is introduced to gather the energy of the beam and illuminate a certain spot, which is a cell, among the whole area. 


% However, to acheive LEO satellite network service, there are some key challanges that need to be resolved. One of the challanges is the unavoidable frequent handover~\cite{38821}. The high speed of LEO satellite forces user equipments (UEs) on the ground to switch the serving satellites frequently, as shown in Figure~\ref{time to ho}. In such scenario, the signalling overhead led by handover signals and the handover interruption time are big issues. 3GPP has discussed some solutions to deal with these issues. By using quasi-earth-fixed cell and satellite switch with re-synchronization~\cite{38300}, the frequent handovers are avoided and the signalling overhead is reduced. For UEs that have already accessed to the network, they can receive the upcoming serving satellite information from the previous one. Also, with the help of the ephemeris data of satellies and the position information of UEs, the matching between satellites and UEs has been largely improved~\cite{38331}. Nontheless, for those UEs who have not access to the network, the random access procedure is the only way for them to get into the network. To establish the connection between satellites and UEs, satellites need to transmit synchronization signal block (SSB) to ground for UEs to capture. Once UEs have successfully receive SSBs, the random access procedure starts and the UEs are able to access to the satellite network. Typically in terrestrial network, ground stations send SSB to the serving area every 20 milliseconds. However, the traditional SSB specifications in LEO satellite communication system do not work because the power budget in LEO satellite is so tight that the power is not enough to send all the serving cells in such a short periodicity. Thus, in this thesis we will find out how to deal with this issue by adjust the SSB periodicity and transmitted power. 

\begin{figure}[h!]
    \centering
    \includegraphics[width=1\textwidth]{figure/time to ho.pdf}
    \caption{Time to handover for min/max cell diameter and varying UE speed}
    \label{time to ho}
\end{figure}